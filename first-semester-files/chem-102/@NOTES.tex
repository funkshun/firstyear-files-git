\documentclass{article}


\title{Notes for Chemistry 102}
\date{Fall 2016}
\author{Boo Fullwood}

\begin{document}

\maketitle
\pagenumbering{gobble}
\newpage

\section{Section}
CHEM 102

\subsection{Day One}
Date = 24/08/16

[CHAPTER 9]
KINETIC MOLECULAR GAS THEORY

Solids and liquids are condensed phases
Phase depends on KE of particles and attraction strength between particles
Temp is the average kinetic energy of a system

Tenets
Gases are composed of molecules in constant motion travelling in straight lines
and only changing directions with collisions

The molecules are very small

The pressure is the force exerted by collisions between the molecules and the
container

All energy is held as kinetic energy

Average energy is proportional to the absolute temperature
\subsection{Day Two}
Date = 26/08/16

CHAPTER 10.1
INTERMOLECULAR FORCES
-Distance must be extremely small

TYPES OF FORCES

{Coulombs Law
Eel = (Q1Q2)/d}

--London Dispersion Forces--
-Dispersion forces arise during instantaneous electron states where the
distribution of electrons is not even. This generates a temporary dipole which
can then induce a dipole in neareby atoms. 
-Larger molecules have larger dispersion forces due to larger electron clouds.
-Most common in Halogens
-Shape matters: large surface area (linear molecules) will have a larger
dispersion area compared to smaller surface area (spherical molecules)

--Dipole-Dipole Interactions--
-Polar molecules are described as having a dipole moment
-Polarity arises due to the existence of polar bonds in a non-symetrical way
{Dipole Moment
mu = delta(r)}
NOTE: Boron execptions

--HYDROGEN BONDS--
-Hydrogen bonded to O, N , or F
-Hydrogen bonding is one of the strongest forces for small molecules
-Hydrogen bonds can bond to negatively charged lone pairs
-Free electrons bond to hydrogen bonds

[DAY THREE]
Date = 29/08/16

KINETICS
-Reaction rates can be determined by monitoring the change in concentration of
either reactants or products as a funtion of time.
-Three types of rate
--Average Rate
--Instantaneous Rate
--Initial Rate
-Apearance and disappearance rate must take into account the stochiometric
relationship between the predicting and reacting molecules
-Unique Rates are determined by the stochimetric ratio
{Unique Rate
\(\frac{1}{a} \times \frac{\Delta A}{\Delta t}\)

    --Factors Affecting Reaction Rate--
-Increasing temperature almost always increases reaction rate
-Increasing concentration of reactant almost always increases reaction rate
-Catalysts lower activation energy and increase reaction rates

[DAY FOUR]
Date = 31/08/16

INTEGRATED RATE LAWS
Zero
At = -kt + A0
First
ln(At) = -kt + ln(A0)
Second
1/At = kt + 1/A0

[DAY FIVE]
Date = 02/09/16

--COLLISION THEORy--
-Gas phase
-A + B | products
-Collision orientation helps determine likelihood of formation
Arrhenius Equations
\(k=Ae^{\frac{-Ea}{R}}\)
f=e^{-Ea/RT}
k=A*f

ln(k1/k2)=(-Ea/R)*((1/T1)-(1/T2))

[Date = 26/09/16]
DETERMINING SPONTANEITY
\(\Delta G = \Delta H - T \Delta S)

\end{document}
