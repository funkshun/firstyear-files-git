\documentclass[12pt,a4paper,titlepage,oneside,]{article}
\usepackage{setspace}
\doublespacing{}
\begin{document}

\title{German Separation and Reunificication}
\author{Boo Fullwood}

\maketitle
\section*{Introduction}
When addressing the topic of crises throughout history, war is the calamity that humanity brings upon itself time and time again. In recent history, the two world wars stand out as crises that shaped the direction in which humanity progressed. In particular, the effects of the second world war on the aggressor nation of Germany shaped its growth from the military nation on the brink of economic collapse in the 1930s to the economic backbone of Europe, despite the required payment of reparations and the harsh treatment of East Germany under the hand of the Soviet Union. This paper will address this seemingly unlikely result, and the factors that facilitated German resilience in the face of the post WWII crisis.
\section*{Initial Separation and Conditions}
Immediately following the end of WWII in Europe, control over Germany was split between the Allied nations, with all of East Germany falling under Soviet control they had born the brunt of casualties in Europe. West Germany was split between the remaining Allies. Prior to the war, economic power was roughly evenly distributed between East and West Germany, though nearly all manufacturing was specialized for armament production and the improvement of the German military. Foreign control over Germany continued until 1990 when the Soviet Union released East Germany and the two nations were reunited. 
\subsection*{Differences Between Soviet and Allied Policies}
Given the nearly 50 years of foreign management, it is unsurprising that the economic identity of Germany has been largely influenced by the policies of the controlling nations during this time. In West Germany, the Allied nations pursued a goal of economic revitalization. It was recognized that the economic distress in Germany had created the perfect environment for a dictator like Hitler to gain power and support. In order to create that a nation that be neither an economic siphon or a future antagonist, Allied nations facilitation the creation of a stable western economy in West Germany, and supported this economy though the payment of reparations and the rebuilding of the German infrastructure which had suffered the most complete destruction of any nation's. The result was a fast recovery and a return to economic independence in a fraction of the time of East Germany. In the East, the Soviet Union tore its reparations from anywhere it could. Some 60 percent of German manufacturing was seized by the Soviet Union, while nearly all of the agricultural land was seized and redistributed with tight restrictions on the amount of land that could be owned by citizens.
\end{document} 
